% \iffalse meta-comment
% Copyright (C) 2015 by Grady Weyenberg <grady.weyenberg@uky.edu>
%
% This work may be distributed and/or modified under the conditions of
% the LaTeX Project Public License, either version 1.3 of this license
% or (at your option) any later version.  The latest version of this
% license is in http://www.latex-project.org/lppl.txt and version 1.3
% or later is part of all distributions of LaTeX version 2005/12/01 or
% later.  
% \fi
%
% \iffalse
%<*internal>
\iffalse
%</internal>
%<*bib>
@unpublished{ukdiss,
	Author = {{University of Kentucky Graduate School}},
	Title = {Electronic Dissertation Preparation Instructions},
	Url = {http://www.gradschool.uky.edu/CurrentStudents/electronic_dissertation_instructions.html},
	Year = {2015}}

@book{latex,
	Address = {Massachusetts},
	Author = {Leslie Lamport},
	Edition = {2nd},
	Publisher = {Addison-Wesley},
	Title = {\LaTeX : A Document Preparation System},
	Year = {1994}}
%</bib>
%<*internal>
\fi
%</internal>
%<*driver>
\ProvidesFile{ukydissertation.dtx}
%</driver>
%<class>\NeedsTeXFormat{LaTeX2e}
%<class>\ProvidesClass{ukydissertation}
%<*class>
[2015/06/05 v0.2 UKY Dissertation Class]
%</class>
%<*driver>
\documentclass{ltxdoc}
\EnableCrossrefs
\begin{document}
  \DocInput{ukydissertation.dtx}
\end{document}
%</driver>
%<*example>
% \fi
% \CheckSum{440}
%
% \title{The \textsf{ukydissertation} class}
% \author{Grady Weyenberg}
% \maketitle
% \section{User Manual}
% This class attempts to provide a method of formatting a dissertation
% to the style guide for the graduate school at the University of
% Kentucky.
% 
% Your \LaTeX{} file should begin with a preamble similar to this:
% \begin{quote}
% \begin{verbatim}
\documentclass[endorse]{ukydissertation}
%%%[...load your packages...]%%
\usepackage{lipsum}
\usepackage{natbib}
\usepackage[numbered,open,openlevel=0]{bookmark}
\bibliographystyle{plainnat}
% \end{verbatim}
% \end{quote}
% You must manually load the |bookmark| package and set the options as
% shown above to get the correct style for the dissertation. This is
% not done in the class definition because you may need to load other
% packages before loading |bookmark|.
% 
% The |endorse| class option causes signatures to appear in the
% appropriate blank spaces in front matter of the
% dissertation. This should be omitted (or replaced with the
% |noendorse| option) until the final stage of the dissertation
% approval process.
%
% \DescribeMacro\title
% \DescribeMacro\author
% \DescribeMacro\department
% \DescribeMacro\college
% \DescribeMacro\director
% \DescribeMacro\dgs
% \DescribeMacro\date
% Next, use these macros define the dissertation document metadata. In
% addition to the usual |title| and |author| macros, there are macros
% for your |department|, |college|, thesis |director| (advisor), and
% the director of graduate studies (|dgs|) for your department.
% Optionally, you may supply a |date|. This is only required if you do
% not want to use the \LaTeX{} default, which is |\today|.
% \begin{quote}
% \begin{verbatim}
\title{A \textsf{ukydissertation} Example Document}
\author{Jeffrey Lebowsky}
\department{Slack}
\college{College of Theosophy}
\director{J.R ``Bob'' Dobbs}
\dgs{Constance M. Dobbs}
\date{19 Prairial CCXXIII}
% \end{verbatim}
% \end{quote}

% \DescribeMacro\abstract
% \DescribeMacro\dedication
% \DescribeMacro\acknowledgements
% \DescribeMacro\keywords
% In addition to the basic metadata, you must also define an
% |abstract| and |keywords|. The |dedication|, and |acknowledgements|
% pages are optional, and will only be added if you supply
% content. 
% \begin{quote}
% \begin{verbatim}
\abstract{This is a example of the ukydissertation class. 
  This class for the \LaTeX{} document system \citep{latex} 
  produces a pdf file conforming to the style guidelines of
  the \citet{ukdiss}.}

\keywords{U of Kentucky, dissertation, latex, documentclass, example}
\dedication{Dedicated to the Flying Spagetti Monster.}
\acknowledgements{\lipsum[1-3]}
% \end{verbatim}
% \end{quote}

% \DescribeMacro\makefrontmatter 
% The first thing in the |document| environment should be the
% |\makefrontmatter| macro. This will use the previous definitions to
% setup the pages prior to the start of the main dissertation
% material. 
% \begin{quote}
% \begin{verbatim}
\begin{document}
\makefrontmatter  
% \end{verbatim}
% \end{quote}

% \DescribeMacro\chapter
% \DescribeMacro\section
% \DescribeMacro\subsection
% \DescribeMacro\subsubsection
% You then proceed to write your dissertation using the classic
% \LaTeX{} commands. The |chapter|, and |section| through
% |subsubsection| environments have been defined, although a 
% |subsubsection| will not be numbered, appear in the ToC, or be bookmarked.
% \begin{quote}
% \begin{verbatim}
\chapter{The First Chapter}
\lipsum[4-10]

\begin{figure}
  \centering
  \framebox[3in][c]{The First Figure}
  \caption[A name for the ToC]{A caption for the first figure.}
  \label{fig:numberone}
\end{figure}

\section[A Shorter Title for ToC]
{The First Subsection Has A Long Title}

\lipsum[11]
\begin{table}
  \caption[A Short Table Title for the ToC]%
  {Tables captions should appear before the beginning of the table.}
  \vspace{\baselineskip}\centering
  \begin{tabular}{ccc}
    \hline
    X & Y & Z\\
    \hline
    1750 & 250 & 56\\
    1760 & 218 & 57\\
    1770 & 204 & 59\\
    \hline
  \end{tabular}
\end{table}
\lipsum[12-15]
% \end{verbatim}
% \end{quote}

% \DescribeMacro\backmatter
% \DescribeMacro\bibliography
% The |\backmatter| macro signals the end of the main text. It is
% similar in effect to \LaTeX's |\appendix|. It should be placed
% before any appendices and the |\bibliography|.
% \begin{quote}
% \begin{verbatim}
\backmatter
\chapter{The First Appendix}
\lipsum[16-17]

\section{First Apppendix section}
\lipsum[18-22]

\bibliography{example}
% \end{verbatim}
% \end{quote}

% \DescribeMacro\vita
% The Vita section is the final section of the dissertation, appearing
% after the bibliography. Use the |\vita| macro to signal the
% beginning of this material.
% \begin{quote}
% \begin{verbatim}
\vita
\lipsum[23-25] 
% \end{verbatim}
% \end{quote}

% And finally, end the document environment.
% \begin{quote}
% \begin{verbatim}
\end{document}
% \end{verbatim}
% \end{quote}

% \iffalse
%</example>
%<*class>
% \fi
% \StopEventually{}
% \section{Implementation}
% \subsection{Initial Code}
% 
% Here we define some switches for use later.
%
% \begin{macro}{\if@endorse}
% Switch indicating if endorsements should be printed in the front
% matter of the thesis.
%    \begin{macrocode}
\newif\if@endorse
%    \end{macrocode}
% \end{macro}

% \begin{macro}{\if@dedication}
% Switch indicating if a dedication section has been supplied.
%    \begin{macrocode}
\newif\if@dedication
\@dedicationfalse
%    \end{macrocode}
% \end{macro}

% \begin{macro}{\if@acknowledgements}
% Switch indicating if an acknowledgements section has been supplied.
%    \begin{macrocode}
\newif\if@acknowledgements
\@acknowledgementsfalse
%    \end{macrocode}
% \end{macro}

% \begin{macro}{\if@mainmatter}
% Indicates whether the current location within the main matter or not.
%    \begin{macrocode}
\newif\if@mainmatter
\@mainmatterfalse
%    \end{macrocode}
% \end{macro}
 
% \subsection{Option Declarations}
% The |endorse|/|noendorse| options enable/disable the printing of
% names in the blank spaces available in the front matter.
%    \begin{macrocode}
\DeclareOption{endorse}{\@endorsetrue}
\DeclareOption{noendorse}{\@endorsefalse}
%    \end{macrocode}
% The |ukydissertation| class is based on the \LaTeX{} |report| class
% on letter paper in a 12pt font.
%    \begin{macrocode}
\DeclareOption*{\PassOptionsToClass{\CurrentOption}{report}}
\ExecuteOptions{noendorse, 12pt, letterpaper}
\ProcessOptions
\LoadClass{report}
%    \end{macrocode}
% Here we load a few packages to help get the page layout and style
% set up correctly.
%    \begin{macrocode}
\RequirePackage[hmargin={1.5in,1in},vmargin=1in,foot=0.5in]{geometry}
\RequirePackage[nottoc]{tocbibind}
\RequirePackage{needspace}
%    \end{macrocode}

% \subsection{Defining Document Metadata}
% These macros define the metadata for the dissertation document.
% \begin{macro}{\college}
% Use to define the name of the college granting the degree.
%    \begin{macrocode}
\newcommand{\college}[1]{\renewcommand{\@college}{#1}}
\newcommand{\@college}{\@latex@warning@no@line%
  {No \noexpand\college given} COLLEGE}
%    \end{macrocode}
% \end{macro}
%
% \begin{macro}{\director}
% Define the name of the committee chair/director (your advisor).
%    \begin{macrocode}
  \newcommand{\director}[1]{\renewcommand{\@director}{#1}}
  \newcommand{\@director}{\@latex@warning@no@line%
    {No \noexpand\director given} Advisor}
%    \end{macrocode}
% \end{macro}
%
% \begin{macro}{\dgs}
% Define the director of graduate studies for your program.
%    \begin{macrocode}
\newcommand{\dgs}[1]{\renewcommand{\@dgs}{#1}}
\newcommand{\@dgs}{\@latex@warning@no@line%
  {No \noexpand\dgs given} DGS}
%    \end{macrocode}
% \end{macro}
%
% \begin{macro}{\department}
% The department granting the degree.
%    \begin{macrocode}
\newcommand{\department}[1]{\renewcommand{\@department}{#1}}
\newcommand{\@department}{\@latex@warning@no@line%
  {No \noexpand\department given} DEPARTMENT}
%    \end{macrocode}
% \end{macro}
%
% \begin{macro}{\keywords}
% Define five keywords for the thesis.
%    \begin{macrocode}
\newcommand{\keywords}[1]{\renewcommand{\@keywords}{#1}}
\newcommand{\@keywords}{\@latex@warning@no@line%
  {No \noexpand\keywords given}}
%    \end{macrocode}
% \end{macro}
%
% \begin{macro}{\abstract}
% Define the dissertation abstract content.
%    \begin{macrocode}
\renewcommand{\abstract}[1]{\renewcommand\@abstract{#1}}
\newcommand{\@abstract}{\@latex@warning@no@line%
  {No \noexpand\abstract given}}
%    \end{macrocode}
% \end{macro}
%
% \begin{macro}{\dedication}
% Define dedication page content. (Optional)
%    \begin{macrocode}
\newcommand{\dedication}[1]{\@dedicationtrue
  \newcommand{\@dedication}{#1}}
%    \end{macrocode}
%
% \end{macro}
% Define acknowledgements page content. (Optional)
% \begin{macro}{\acknowledgements}
%    \begin{macrocode}
\newcommand{\acknowledgements}[1]{\@acknowledgementstrue
  \newcommand{\@acknowledgements}{#1}}
%    \end{macrocode}
% \end{macro}
%
% \subsection{Document Control Macros}
% \begin{macro}{\makefrontmatter}
% This command replaces the |\maketitle| macro.
%    \begin{macrocode}
\newcommand\makefrontmatter{%
  \maketitle \@makeabstract \@makeapproval 
  \if@dedication\@makededication\fi
  \setcounter{page}{3}
  \pagestyle{plain}
  \if@acknowledgements\@makeacknowledgements\fi
  \pdfbookmark[chapter]{\contentsname}{toc}
  \tableofcontents \listoftables \listoffigures
  \clearpage
  \pagenumbering{arabic} \global\let\thanks\relax
  \global\let\maketitle\relax \global\let\title\relax
  \global\let\author\relax \global\let\date\relax
  \global\let\and\relax \global\let\abstract\relax
  \global\let\dedication\relax}
%    \end{macrocode}
% \end{macro}
%    
% \begin{macro}{\backmatter}
% This macro replaces the |\appendix| macro.
%    \begin{macrocode}
\newcommand{\backmatter}{\@mainmatterfalse\appendix}
%    \end{macrocode}
% \end{macro}
% 
% \begin{macro}{\vita}
% This macro begins the vita section of the thesis.
%    \begin{macrocode}
\newcommand{\vita}{\clearpage\chapter*{Vita}%
  \addcontentsline{toc}{chapter}{Vita}}
%    \end{macrocode}
% \end{macro}
%
% \subsection{Section header definitions}
%    \begin{macrocode}
\renewcommand{\@makechapterhead}[1]{%
  {\parindent \z@ \centering \normalfont
    \ifnum \c@secnumdepth >\m@ne
    \bfseries \@chapapp\space \thechapter
        \par\nobreak
        \vskip 10\p@
    \fi
    \interlinepenalty\@M
    \bfseries #1\par\nobreak
    \vskip 20\p@
  }}
\renewcommand{\@makeschapterhead}[1]{%
  {\parindent \z@ \raggedright
    \normalfont
    \interlinepenalty\@M
    \centering\bfseries  #1\par\nobreak
    \vskip 20\p@
  }}
\renewcommand{\section}{%
  \@startsection {section}{1}{\z@}%
  {-3.5ex \@plus -1ex \@minus -.2ex}%
  {2.3ex \@plus.2ex}%
  {\normalfont\bfseries}}
\renewcommand{\subsection}{%
  \@startsection{subsection}{2}{\z@}%
  {-3.25ex\@plus -1ex \@minus -.2ex}%
  {1.5ex \@plus .2ex}%
  {\normalfont\bfseries}}
\newcommand{\@period}[1]{#1.--- }
\renewcommand{\subsubsection}{%
  \@startsection{subsubsection}{3}{\z@}%
  {3.25ex \@plus1ex \@minus.2ex}%
  {\z@}%
  {\normalfont\normalsize\itshape\@period}}
%    \end{macrocode}

% \subsection{Front Matter Layouts}
% Setup the page layouts for the title, abstract, approval,
% dedication, acknowledgements, table of contents and list of figures/tables pages. 
%    \begin{macrocode}
\renewcommand\maketitle{%
  \setcounter{page}{0}
  \pagestyle{empty}
  {\let\footnotesize\small
  \let\footnoterule\relax
  \let\footnote\thanks
  \null
  \bookmark[named=FirstPage,level=-1]{Title Page}
  \vskip 1.5in%
  \begin{center}%  
  \parbox[t][1.5in]{\textwidth}%
  {\centering\MakeUppercase\@title\par}%
  \par
  \parbox[t][3\baselineskip]%
  {0.5\textwidth}{\hrule\vskip\baselineskip%
    \centering DISSERTATION%
    \par\vskip\baselineskip\hrule}
  \vskip\baselineskip%
  {A dissertation submitted in partial fulfillment of the \\
    requirements for the degree of Doctor of Philosophy in the \\
    \@college\\ at the University of Kentucky\par}%
  \vskip 2\baselineskip%
    {By\par\vskip 0.5\baselineskip\@author\par}%
    \vskip 0.5\baselineskip%
    {Lexington, Kentucky\par}%
    \vskip 1em%
    {Director: Dr. \@director, Professor of \@department\par}%
    \vskip 0.5\baselineskip%
    {Lexington, Kentucky\par}%
    \vfill
    {Copyright \textcopyright\ \@author\ \the\year\par}%
    \vspace*{1.5in}\par
  \end{center}\par
  \null}
  \clearpage
  \setcounter{footnote}{0}}

\newcommand\@makeabstract{%
  \pagestyle{empty}
  \pagenumbering{Roman}
  \hypertarget{abstract}{}
\bookmark[dest=abstract,level=0]{Abstract}
\null\vskip 1.5in
\begin{center}%
  \parbox[t][1in]{\textwidth}{\centering ABSTRACT OF DISSERTATION\par}
  \MakeUppercase\@title\par
\end{center}
\@abstract
\par\vspace\baselineskip\noindent 
  KEYWORDS: \parbox[t]{0.8\textwidth}{\@keywords}
  \par\needspace{1.5in}\null\vfill
  \hfill\parbox[t][1.5in]{0.5\textwidth}{%
    \if@endorse
    {\centering\MakeUppercase\@author\par}\vskip 2pt
    \else\fi
    \hrule
    \vskip 2pt
    Student's Signature\par
    \vfil
    \if@endorse
    {\centering\MakeUppercase\@date\par}\vskip 2pt
    \else\fi
    \hrule
  \vskip 2pt
  Date\par}
  \par\null\clearpage}

\newcommand\@makeapproval{
  \pagenumbering{roman}
  \null\vskip 2in
  \noindent\parbox[t][1in]{\textwidth}%
  {\centering\MakeUppercase\@title\par}\par
  \begin{centering}
    By
    \vskip0.5\baselineskip
    \@author\par
  \end{centering}
  \vfill 
  \hfill\parbox[t][3in]{0.5\textwidth}{%
    \if@endorse
    {\centering\MakeUppercase\@director\par}\vskip 2pt
    \else\fi
    \hrule
    \vskip 2pt
    Director of Dissertation\par
    \vfil
    \if@endorse
    {\centering\MakeUppercase\@dgs\par}\vskip 2pt
    \else\fi
    \hrule
    \vskip 2pt
    Director of Graduate Studies\par
    \vfil
    \if@endorse
    {\centering\MakeUppercase\@date\par}\vskip 2pt
    \else\fi
    \hrule
  \vskip 2pt
  Date\par}\hskip 1in
    \par\vfil\clearpage}


\newcommand\@makeacknowledgements{
  \phantomsection
  \addcontentsline{toc}{chapter}{Acknowledgements}
  \null
  {\centering ACKNOWLEDGEMENTS \par}
  \vspace{\baselineskip}\noindent
  \@acknowledgements
  \par\vfil\clearpage}

\newcommand\@makededication{%
  \null\vfil
  \@dedication
  \par\vfil
  \clearpage}

\renewcommand{\contentsname}{Table of Contents}
%    \end{macrocode}
%
% \Finale
% \iffalse
%</class>
% \fi